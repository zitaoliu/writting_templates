%%%%%%%%%%%%%%%%%%%%%%%%%%  ltexpprt.tex  %%%%%%%%%%%%%%%%%%%%%%%%%%%%%%%%
%
% This is ltexpprt.tex, an example file for use with the SIAM LaTeX2E
% Preprint Series macros. It is designed to provide double-column output.
% Please take the time to read the following comments, as they document
% how to use these macros. This file can be composed and printed out for
% use as sample output.

% Any comments or questions regarding these macros should be directed to:
%
%                 Donna Witzleben
%                 SIAM
%                 3600 University City Science Center
%                 Philadelphia, PA 19104-2688
%                 USA
%                 Telephone: (215) 382-9800
%                 Fax: (215) 386-7999
%                 e-mail: witzleben@siam.org


% This file is to be used as an example for style only. It should not be read
% for content.

%%%%%%%%%%%%%%% PLEASE NOTE THE FOLLOWING STYLE RESTRICTIONS %%%%%%%%%%%%%%%

%%  1. There are no new tags.  Existing LaTeX tags have been formatted to match
%%     the Preprint series style.
%%
%%  2. You must use \cite in the text to mark your reference citations and
%%     \bibitem in the listing of references at the end of your chapter. See
%%     the examples in the following file. If you are using BibTeX, please
%%     supply the bst file with the manuscript file.
%%
%%  3. This macro is set up for two levels of headings (\section and
%%     \subsection). The macro will automatically number the headings for you.
%%
%%  5. No running heads are to be used for this volume.
%%
%%  6. Theorems, Lemmas, Definitions, etc. are to be double numbered,
%%     indicating the section and the occurence of that element
%%     within that section. (For example, the first theorem in the second
%%     section would be numbered 2.1. The macro will
%%     automatically do the numbering for you.
%%
%%  7. Figures, equations, and tables must be single-numbered.
%%     Use existing LaTeX tags for these elements.
%%     Numbering will be done automatically.
%%
%%
%%%%%%%%%%%%%%%%%%%%%%%%%%%%%%%%%%%%%%%%%%%%%%%%%%%%%%%%%%%%%%%%%%%%%%%%%%%%%%%


\PassOptionsToPackage{english}{babel}
\documentclass[twoside,leqno,twocolumn]{article}
\usepackage{ltexpprt}

\usepackage{amsmath}
\usepackage{amsfonts}
\usepackage{amssymb}

\usepackage{url}
\usepackage{graphicx}

\usepackage{algorithm}
\usepackage{algorithmic}
\usepackage{paralist}

\usepackage{booktabs}

\DeclareMathOperator{\vect}{vec}

\begin{document}


%\setcounter{chapter}{2} % If you are doing your chapter as chapter one,
%\setcounter{section}{3} % comment these two lines out.

\title{\Large SDM Title}
\author{Jerry\thanks{Pinterest, San Francisco CA USA. Email: abc@gmail.com} \\
\and
Zitao Liu$^*$}
\date{}

\maketitle


%\pagenumbering{arabic}
%\setcounter{page}{1}%Leave this line commented out.

\begin{abstract} \small\baselineskip=9pt 
Lorem ipsum dolor sit amet, consectetur adipisicing elit, sed do eiusmod
tempor incididunt ut labore et dolore magna aliqua. Ut enim ad minim veniam,
quis nostrud exercitation ullamco laboris nisi ut aliquip ex ea commodo
consequat. Duis aute irure dolor in reprehenderit in voluptate velit esse
cillum dolore eu fugiat nulla pariatur. Excepteur sint occaecat cupidatat non
proident, sunt in culpa qui officia deserunt mollit anim id est laborum.
\end{abstract}


\section{Introduction}
\label{sec:intro}

Lorem ipsum dolor sit amet, consectetur adipisicing elit, sed do eiusmod
tempor incididunt ut labore et dolore magna aliqua. Ut enim ad minim veniam,
quis nostrud exercitation ullamco laboris nisi ut aliquip ex ea commodo
consequat. Duis aute irure dolor in reprehenderit in voluptate velit esse
cillum dolore eu fugiat nulla pariatur. Excepteur sint occaecat cupidatat non
proident, sunt in culpa qui officia deserunt mollit anim id est laborum. \cite{liu2015regularized}.

\begin{algorithm}[!bpht]
\small \caption{Caption.}
\label{alg:summary}
{INPUT:}
\begin{compactitem}
\item Some input
\end{compactitem}
{PROCEDURE:}
\begin{algorithmic}[1]
\STATE // Comment
\REPEAT
   \STATE Update 
   \FOR{m: 1 $\rightarrow$ N}
   		\STATE Update.
   		\FOR{t: 2 $\rightarrow$ $T_m-1$}
   				\STATE Update.
   		\ENDFOR
   		\STATE Update.
   \ENDFOR
\UNTIL{Convergence}
\STATE // Optimize
\STATE Compute.
\end{algorithmic}
{OUTPUT:}
\begin{compactitem}
\item Some output.
\end{compactitem}
\renewcommand*\arraystretch{1.0}
\end{algorithm}



\begin{theorem}
\label{thm:lip}
Lorem ipsum dolor sit amet, consectetur adipisicing elit, sed do eiusmod
tempor incididunt ut labore et dolore magna aliqua. Ut enim ad minim veniam,
quis nostrud exercitation ullamco laboris nisi ut aliquip ex ea commodo
consequat. Duis aute irure dolor in reprehenderit in voluptate velit esse
cillum dolore eu fugiat nulla pariatur. Excepteur sint occaecat cupidatat non
proident, sunt in culpa qui officia deserunt mollit anim id est laborum.
\end{theorem}

\begin{proof}
The proof of this theorem appears in the supplemental material.
\end{proof}

\begin{figure}[!htbp]
\centering
%\includegraphics[width=0.45\textwidth]{fig/fig.pdf}
\caption{Predictions for flour price series in Buffalo.}
\label{fig:flour_buffalo}
\end{figure}



\begin{table}[!hptb]
\caption{Average-MAPE results on \emph{clinical} dataset.}
\centering
\begin{tabular}{@{}lccccc@{}} \toprule
\label{tab:clinical_mape}
  \#  of states & 10 & 15  & 20 & 25 & 30   \\ \midrule
Spectral & 6.29 & 6.24 & 6.32 & 6.04 & 6.00  \\
EM & 3.97 & 3.54 & 3.54 & 3.53 & 3.53  \\
gLDS-ridge & 3.22 & 3.21 & 3.21 & 3.21 & 3.21  \\
gLDS-smooth & \textbf{3.21} & \textbf{3.20} & \textbf{3.20} & \textbf{3.19} & \textbf{3.19}  \\ \bottomrule
\end{tabular}
\end{table}


\bibliographystyle{siam}
\bibliography{bib/sdm}


\end{document}

% End of ltexpprt.tex